% \section{Related Works} \label{related}
% A variety of works have looked into various methods of rigid body dynamics for simulating closed-loop linkages, how to address the numerical issues of redundant constraints that arise in these problems, as well as how joint clearance can effect these linkage systems. 

% \subsection{Redundant Constraints}

% Muller's work explored the concept of generic degrees of freedom (DOF) in mechanisms, showing how imperfections in link geometry impact mobility \cite{muller_generic_2009}. The author demonstrated that most mechanisms are not overconstrained and have predictable DOFs, making it easier to classify mechanisms as overconstrained, underconstrained, or kinematotropic. Wojtyra et al.,  provided analytic conditions for determining uniquely solvable reaction forces in systems with redundant nonholonomic constraints \cite{wojtyra_joint_2009}. They developed numerical methods to detect constraints that allow for unique solutions, contributing to a more accurate modeling of constrained multibody systems. Wojtyra et al., later presented a mathematical framework for identifying and calculating joint reactions in mechanisms with dependent constraints due to redundancy or singular configurations. It showed how specific joint reactions could be uniquely determined despite the overall indeterminacy of the system \cite{wojtyra_solvability_2013}. Pekal et al., compared several numerical approaches—SVD, QR, and nullspace methods—and demonstrated how to check the uniqueness of selected forces within multibody systems efficiently \cite{pekal_constraint-matrix-based_2023}.

% \subsection{Joint Clearance Analysis}

% Funabashi et al. systematically analyzed the dynamic behavior of multilink mechanisms with joint clearances, incorporating elastic deformations and friction. The paper revealed how clearances and crank speeds affect the relative motion, input torque, and output displacement of mechanisms \cite{noauthor_dynamic_nodate}. Soong et al. developed an analytical model to predict the dynamic behavior of a slider-crank mechanism with radial clearance in its joints \cite{soong_theoretical_1990}. Akhadkar et al. analyzed the impact of joint clearances on the performance of multibody systems by modeling the contact forces in pin and hole assemblies. It demonstrated how even small clearances degrade the performance of mechanisms, particularly at lower input speeds \cite{Akhadkar_influence_2014}. Tan et al. developed a continuous analysis method to investigate dynamic responses in planar mechanisms with clearance joints, focusing on how joint contact and friction influence the system's motion \cite{tan_continuous_2017}. 

% Mutawe et al. addressed path generation for four-bar linkages under joint clearance constraints, incorporating ANSI tolerance standards. The authors developed a design synthesis method that accounted for joint tolerances, ensuring that mechanisms could achieve specified paths within defined tolerance limits\cite{mutawe_designing_2012}. Qi et al. proposed a method to introduce joint clearances in overconstrained linkages to prevent jamming due to component deformation. Their model optimizes clearance placement to ensure smooth mechanism operation under deformation \cite{qi_synthesis_2023}.

% \subsection{Jamming and Failure}
% Dupont et al. explored the relationship between Coulomb friction and jamming in rigid-body dynamics. They provided conditions for jamming and wedging in constrained systems, illustrating their findings with practical examples \cite{dupont_jamming_1994}. Rivera et al., reviewed failures in spacecraft deployable systems, particularly focusing on solar arrays and antennas. The analysis identified common causes of deployment failures and suggested best practices for future designs to mitigate the risks of jamming and other deployment issues \cite{rivera_study_nodate}.

